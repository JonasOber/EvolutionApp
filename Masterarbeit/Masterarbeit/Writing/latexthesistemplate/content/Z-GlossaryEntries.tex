% !TeX encoding=utf8
% !TeX spellcheck = en-US

%%% --- Acronym definitions
\IfDefined{newacronym}{%
% place these definitions before \begin{document}
\newacronym{NA}{NA}{numerical Apertur}
\newacronym{DOF}{DOF}{depth of field}
\newacronym{PSF}{PSF}{point spread function}
}%


%%% --- Symbol list entries
\IfDefined{newglossaryentry}{%
% place these definitions before \begin{document}
\newglossaryentry{symb:Pi}{%
	name=$\pi$,%
	description={mathematical constant},%
	sort=symbolpi, type=symbolslist%
}
\newglossaryentry{symb:Phi}{%
	name=$\varphi$,%
	description={arbitrary angle},%
	sort=symbolphi, type=symbolslist%
}
\newglossaryentry{symb:Lambda}{%
	name=$\lambda$,%
	description={wavelength},%
	sort=symbollambda, type=symbolslist%
}	
}%

%%% --- Glossary entries

% place these definitions before \begin{document}
\IfDefined{newglossaryentry}{%
\newglossaryentry{glos:CD}{name=Compact disc (CD),
	description={The Compact Disc (also known as a CD) is an optical disc used
		to store digital data. It was originally developed to store and playback sound
		recordings exclusively, but later expanded to encompass storage of data (Source:
		wikipedia)}
}%
\newglossaryentry{glos:DVD}{name=DVD,
	description={DVD is an optical disc storage media format, invented and
		developed by Philips, Sony, Toshiba, and Panasonic in 1995. DVDs offer
		higher storage capacity than Compact Discs while having the same dimensions.
		The basis of the DVD name stems from the term \textit{digital versatile disc}.
		(Source: wikipedia)}
}%
}%
